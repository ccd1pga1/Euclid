\begin{post}
Let the following be postulated: to draw a straight line from any point to any point.
\end{post}

Although it is a direct translation from the Greek, it would be more accurate to say 'from every point to every point', just like in $\textit{Postulate 3}$, where the first words are 'with every centre and distance'.

The first idea presented here suggests that whenever we have two points, like $A$ and $B$, we can always draw a straight line connecting them We denoted this as $\overline{AB}$. This construction uses a straightedge, as does another method discussed in a later postulate.

\sidenote{\raggedright\ssmall{The diagram illustrates a straight line drawn from point A to point B, with both points highlighted in red.}}
\begin{figure}[H]
\centering
	\begin{tikzpicture}
		\tkzDefPoint(0,0){A}
		\tkzDefPoint(4,0){B}
		\tkzDrawSegment(A,B)
		\tkzDrawPoints[red](A,B)
		\tkzLabelPoints[above](A,B)
	\end{tikzpicture}
	\caption{The line}
\end{figure}

Even though it's not directly stated, there's only one straight line possible between these two points. Euclid assumes this uniqueness as part of the $\textit{Postulate}$.This implicit assumption of uniqueness underpins much of Euclidean geometry, reinforcing the foundational nature of $\textit{Postulate 1}$, though it would have been clearer if he would have mentioned it explicitly.However, even Proclus agrees, based on further writings of Euclid, such as $\textit{Proposition I.4}$, that there must only be one straight line from any point to any point. Proclus elaborates on this in his commentary by referencing Euclid's $\textit{Proposition I.4}$, where the construction of congruent triangles presupposes the uniqueness of the line connecting two points.

In the latter part of Euclid's Elements, dealing with solid geometry, the two points mentioned in the $\textit{Postulate}$ can be any pair in space. $\textit{Proposition XI.1}$ asserts that if a part of a line lies within a plane, then the entirety of the line does as well. In the sections on plane geometry, it is implied that $\overline{AB}$, connecting points $A$ and $B$, lies within the plane under discussion. This extension of principles from plane to solid geometry underscores Euclid’s systematic approach, demonstrating that fundamental truths apply irrespective of the spatial context.

\clearpage