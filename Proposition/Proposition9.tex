
\section*{Proposition 9}

\renewcommand\qedsymbol{Q.E.F}

\begin{con}
It is required to bisect $\angle{BAC}$.
\end{con}

\begin{figure}[H]
 	\begin{tikzpicture}
 		\tkzDefPoint(0,0){A}
 		\tkzDefPoint(-1.5,-1.5){B}
 		\tkzDefPoint(1.5,-1.5){C}
 		\tkzDrawSegments(A,C A,B)
 		\tkzLabelPoints(A,B,C)
 		\tkzDrawPoints[gray](A,B,C)
	\end{tikzpicture}
 	\caption{}
\end{figure}

\textbf{Proof by Construction:}

\begin{enumerate}
    \item Take an arbitrary point $D$ on $\overline{AB}$.
    \item Cut off $\overline{AE}$ from $\overline{AC}$ equal to $\overline{AD}$, and join $\overline{DE}$.\hfill \textcolor{red}{I.3}
 
\begin{figure}[H]   
	\begin{tikzpicture}
 		\tkzDefPoint(0,0){A}
 		\tkzDefPoint(-3,-3){B}
 		\tkzDefPoint(3,-3){C}
 		\tkzDefPointOnLine[pos=0.6](A,B)	\tkzGetPoint{D}
 		\tkzDefPointOnLine[pos=0.6](A,C)	\tkzGetPoint{E}
 		\tkzDrawSegments(A,C A,B D,E)
 		\tkzLabelPoints(A,B,C,D,E)
 		\tkzDrawPoints[gray](A,B,C,D,E)
	\end{tikzpicture} 
    	\caption{}
\end{figure}

\clearpage
    
    \item Construct the equilateral $\triangle{DEF}$ on $DE$.\hfill \textcolor{red}{I.1}

\begin{figure}[H]    
	\begin{tikzpicture}
 		\tkzDefPoint(0,0){A}
 		\tkzDefPoint(-3,-3){B}
 		\tkzDefPoint(3,-3){C}
 		\tkzDefPointOnLine[pos=0.6](A,B)	\tkzGetPoint{D}
 		\tkzDefPointOnLine[pos=0.6](A,C)	\tkzGetPoint{E}
 		\tkzDrawSegments(A,C A,B D,E)
 		\tkzInterCC(D,E)(E,D)			\tkzGetPoints{G}{F}
 		\tkzDrawPolygon[fill=green, opacity=0.5](D,F,E)
 		\tkzLabelPoints(A,B,C,D,E,F)
 		\tkzDrawPoints[gray](A,B,C,D,E,F)
	\end{tikzpicture} 
    	\caption{}
\end{figure}

    \item Join $AF$.
    \item Then, the $\angle{BAC}$ is bisected by the straight line $\overline{AF}$.

\begin{figure}[H]    
	\begin{tikzpicture}
 		\tkzDefPoint(0,0){A}
 		\tkzDefPoint(-3,-3){B}
 		\tkzDefPoint(3,-3){C}
 		\tkzDefPointOnLine[pos=0.6](A,B)	\tkzGetPoint{D}
 		\tkzDefPointOnLine[pos=0.6](A,C)	\tkzGetPoint{E}
 		\tkzDrawSegments(A,C A,B D,E)
 		\tkzInterCC(D,E)(E,D)			\tkzGetPoints{G}{F}
 		\tkzDrawPolygon(D,F,E)
 		\tkzLabelPoints(A,B,C,D,E,F)
 		\tkzDrawPoints[gray](A,B,C,D,E,F)
 		\tkzDrawSegment[red](A,F)
 	\end{tikzpicture}   
    	\caption{}
\end{figure}

\end{enumerate}

\clearpage

\begin{proof}

\begin{lemma}
since $AD = AE$ and $AF$ is common, the two sides $AD$ and $AF$ are equal to the two sides $EA$ and $AF$ respectively.\hfill \textcolor{red}{C.N.1,I.Post.1}
\end{lemma}

\begin{lemma}
Since $\overline{DF} = \overline{EF}$, the angles $\angle{DAF}$ and $\angle{EAF}$ are equal.\hfill \textcolor{red}{Def.20,I.8}
\end{lemma}

\textbf{Conclusion:} 

\begin{itemize}

\item[$\therefore$] the given rectilinear angle $\angle BAC$ is bisected by the straight line $AF$.\hfill \textcolor{red}{I.8}

\end{itemize}

\end{proof}

\clearpage
