
\section*{Proposition 45}

\renewcommand\qedsymbol{Q.E.F}

\begin{con}
To construct a parallelogram equal to a given rectilinear figure $ABCD$ in a given rectilinear angle $E$.
\end{con}

\begin{figure}[H]
	\begin{subfigure}{0.3\textwidth}
		\begin{tikzpicture}
			\tkzDefPoint(0,0){B}
			\tkzDefPoint(-1,2.8){A}
			\tkzDefPoint(0,3){D}
			\tkzDefPoint(1,2.4){C}
			\tkzLabelPoints[below](B)
			\tkzLabelPoints[left](A)
			\tkzLabelPoints[right](C)
			\tkzLabelPoints[above](D)
			\tkzDrawPoints(A,B,C,D)	
			\tkzDrawSegments(A,B B,C C,D D,A B,D)
		\end{tikzpicture}
		\caption{Rectilier shape $ABCD$}
	\end{subfigure}
%%%%%%%%%%%%%%%%%%%%%%%
	\begin{subfigure}{0.3\textwidth}
		\begin{tikzpicture}
			\tkzDefPoint(2,0){X}
			\tkzDefShiftPoint[X](0:1){Y}
			\tkzDefShiftPoint[X](80:1){Z}
			\tkzDefPoint(2.5,0.5){E}
			\tkzDrawSegments(X,Y X,Z)
			\tkzDrawPoints(X,Y,Z)
			\tkzLabelPoints(E)
		\end{tikzpicture}
		\caption{Rectiliear angle $E$}
	\end{subfigure}
	\caption{}
\end{figure}

\textbf{Given:}
\begin{itemize}
    \item Rectilinear figure $ABCD$.
    \item Rectilinear angle $E$.
     \item A straight line can be drawn between any two points, join $\overline{BD}$.\hfill\textcolor{red}{ I.Post.1}
\end{itemize}

\begin{con}

\begin{enumerate}

\item[]

    \item A straight line can be drawn between any two points, join $\overline{BD}$.\hfill\textcolor{red}{ I.Post.1}
    
      \item Construct the parallelogram $FGHK$ equal $\widetriangle{ABD}$ in $\angle{HKF}$ which equals $E$.\hfill\textcolor{red}{I.42}
      
      \clearpage
      
    \item Construct the parallelogram $GLMN$ equal to $\widetriangle{DBC}$ to $\overline{GH}$ in $\angle{GHM}$ which equals $E$.\hfill\textcolor{red}{I.44}

\end{enumerate}

\begin{figure}[H]
	\begin{subfigure}{0.35\textwidth}
		\begin{tikzpicture}
		\tkzDefPoint(0,0){K}
		\tkzDefPoint(1,3){F}
		\tkzDefPoint(2,0){H}
		\tkzDefPoint(3,3){G}
		\tkzLabelPoints[left](K,F)	
		\tkzLabelPoints[below](H)
		\tkzLabelPoints[above](G)
		\tkzDrawPoints(K,H,G,F)
		\tkzDrawPolygon[fill=yellow, opacity=0.3,thick](K,H,G,F)
		\end{tikzpicture}
		\caption{}
	\end{subfigure}
%%%%%%%%%%%%%
	\begin{subfigure}{0.35\textwidth}
		\begin{tikzpicture}
			\tkzDefPoint(0,0){H}
			\tkzDefPoint(1,0){M}
			\tkzDefPoint(1,3){G}
			\tkzDefPoint(2,3){L}
			\tkzLabelPoints[right](L,M)
			\tkzLabelPoints[above](G)
			\tkzLabelPoints[below](H)
			\tkzDrawPoints(H,M,L,G)
			\tkzDrawPolygon[fill=purple, opacity=0.3,thick](H,M,L,G)
		\end{tikzpicture}
		\caption{}
	\end{subfigure}
	\caption{}
\end{figure}

\begin{figure}[H]
	\begin{tikzpicture}
		\tkzDefPoint(0,0){K}
		\tkzDefPoint(1,3){F}	
		\tkzDefPoint(2,0){H}	
		\tkzDefPoint(3,3){G}	
		\tkzDefPoint(3,0){M}	
		\tkzDefPoint(4,3){L}
		\tkzLabelPoints[left](K,F)	
		\tkzLabelPoints[below](H)
		\tkzLabelPoints[above](G)
		\tkzLabelPoints[right](L,M)
		\tkzDrawPoints(K,H,G,F)
		\tkzDrawPolygon[fill=yellow, opacity=0.3,thick](K,H,G,F)	
		\tkzDrawPolygon[fill=purple, opacity=0.3,thick](H,M,L,G)	
	\end{tikzpicture}
	\caption{}
\end{figure}

\end{con}

\begin{proof}
\begin{enumerate}
    \item Since angle $E$ equals both $\angle{HKF}$ and $\angle{GHM}$, $\angle{HKF}$ equals $\angle{GHM}$.\hfill\textcolor{red}{ C.N.1}
    
    \item Add $\angle{KHG}$ to both sides. 
    
    \item[$\therefore$] 
    \[\angle{FKH} + \angle{KHG} = \angle{KHG} + \angle{GHM}\]\hfill\textcolor{red}{ C.N.2}
    
   \item \[\angle{FKH} + \angle{KHG} 2\times\ang{90} \]
   
   \item[$\therefore$]
   \[\angle{KHG} + \angle{GHM} =2\times\ang{90}\]\hfill\textcolor{red}{ I.29, C.N.1}
 
 \clearpage
   
    \item With a straight line $\overline{GH}$, and at point $H$ on it, two straight lines $\overline{KH}$ and $\overline{HM}$ not lying on the same side make the adjacent angles together equal to two right angles. 
    
    \item[$\therefore$] $\overline{KH}$ is in a straight line with $\overline{HM}$.\hfill\textcolor{red}{ I.29}
    
    \item Since line $\overline{HG}$ falls upon the parallels $\overline{KM}$ and $\overline{FG}$, the alternate $\angle{MHG}$ and $\angle{HGF}$ are equal.\hfill\textcolor{red}{ I.29}
    
  \item Add $\angle{HGL}$ to both sides,
  \[\angle{MHG} + \angle{HGL} = \angle{HGF} + \angle{HGL}\]\hfill\textcolor{red}{C.N.2}
  
  \item \[\angle{MHG} + \angle{HGL} = 2\times\ang{90}\]
  
  \item[$\therefore$]
   \[\angle{HGF} + \angle{HGL} =2\times\ang{90}\]\hfill\textcolor{red}{ I.29, C.N.1, I.14}
   
    \item $\overline{FG}$ is in a straight line with $\overline{GL}$.\hfill\textcolor{red}{ I.34, I.30}
    
    \item Since \[\overline{FK} \eqparallel \overline{HG}\] 
    and 
    \[\overline{HG} \eqparallel \overline{ML}\]
    
    \item[$\therefore$] 
    \[\overline{KF} \eqparallel \overline{ML}\]
    and the straight lines $\overline{KM}$ and $\overline{FL}$ join them at their ends. 
    
    \clearpage
    
    \item[$\therefore$] \[\overline{KM} \eqparallel \overline{FL}\]\hfill\textcolor{red}{ I.33, C.N.1}
    
   \item Since $\widetriangle{ABD}$ equals parallelogram $FKHG$, and $\widetriangle{DBC}$ equals parallelogram $GHML$, the whole rectilinear figure $ABCD$ equals the whole parallelogram $KFLM$.\hfill\textcolor{red}{ C.N.2}
   
    \item[$\therefore$] the parallelogram $KFLM$ has been constructed equal to the given rectilinear figure $ABCD$ in the$\angle{FKM},$ which equals the given angle $E$.
    
\end{enumerate}

\end{proof}

\clearpage
