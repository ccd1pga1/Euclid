
\renewcommand\qedsymbol{Q.E.F}

\section*{Proposition 23}

\begin{con}
To construct a rectilinear angle equal to a given rectilinear angle on a given straight line and at a point on it, let $\angle{DCE}$ be the given rectilinear angle, $\overline{AB}$ the given straight line, and $A$ the point on it. Construct a rectilinear angle $\angle{KAG}$ on the given straight line $\overline{AB}$ and at the point $A$ on it such that $\angle{DCE} = \angle{FAG}$.
\end{con}

\textbf{Construction:}

\begin{figure}[H]
	\begin{subfigure}{0.45\textwidth}
		\begin{tikzpicture}
			\tkzDefPoint(0,0){C}
			\tkzDefPoint(2.5,-2){Y}
			\tkzDefPoint(2,1.5){Z}
			\tkzDefPointOnLine[pos=0.75](C,Y)	\tkzGetPoint{D}
			\tkzDefPointOnLine[pos=0.75](C,Z)	\tkzGetPoint{E}
			\tkzDrawSegments(Y,C C,Z)
			\tkzLabelPoints(C,D,E)
			\tkzDrawPoints(C,D,E)
		\end{tikzpicture}
		\caption{}
	\end{subfigure}
%%%%%%%%%%%%%%%%5
	\begin{subfigure}{0.45\textwidth}
		\begin{tikzpicture}
		\tkzDefPoint(0,0){A}
		\tkzDefPoint(6,0){B}
		\tkzDrawSegment(A,B)
		\tkzDrawPoints(A)
		\tkzLabelPoints(A,B)
		\end{tikzpicture}
		\caption{}
	\end{subfigure}
	\caption{}
\end{figure}

\begin{enumerate}

	\item Draw the segment $\overline{ED}$. \hfill\textcolor{red}{Post.1}
	
\begin{figure}[H]
	\begin{tikzpicture}
		% Def
		\tkzDefPoint(0,0){C}
		\tkzDefPoint(2.5,-2){Y}
		\tkzDefPoint(2,1.5){Z}
		\tkzDefPointOnLine[pos=0.75](C,Y)	\tkzGetPoint{D}
		\tkzDefPointOnLine[pos=0.75](C,Z)	\tkzGetPoint{E}
		% draw
		\tkzDrawSegments(Y,C C,Z)
		\tkzDrawSegment[red](E,D)
		\tkzDrawPoints(C,D,E)
		% label
		\tkzLabelPoints(C,D,E)
	\end{tikzpicture}
	\caption{}
\end{figure}

    	\item Draw the given straight line $\overline{AB}$ and mark the point $A$.
     
     	\item On $\overline{AB}$ mark the points $G$, $H$ and $I$ so that,  
     	\[\overline{AG}=\overline{CD}\]
     	 \[\overline{GH}=\overline{ED}\]
     	 \[\overline{AI}=\overline{CE}\] \hfill\textcolor{red}{I.22}
     
\begin{figure}[H]
		\begin{tikzpicture}
		% def
\tkzDefPoint(0,0){X}
\tkzDefPoint(2.5,-2){Y}
\tkzDefPoint(2,1.5){Z}
\tkzDefPoint(0,-5){A}
\tkzDefPoint(10,-5){B}
% calc
\tkzCalcLength(X,Y) \tkzGetLength{XY}
\tkzCalcLength(X,Z) \tkzGetLength{XZ}
\tkzCalcLength(Y,Z) \tkzGetLength{YZ}
\tkzDefPointWith[linear normed,K=\XY](A,B) \tkzGetPoint{G}
\tkzDefPointWith[linear normed,K=\YZ](G,B) \tkzGetPoint{H}
\tkzDefPointWith[linear normed,K=\XZ](A,B) \tkzGetPoint{I}
\tkzCalcLength(A,I) \tkzGetLength{AI}
\tkzCalcLength(H,I) \tkzGetLength{HI}
\tkzCalcLength(A,G) \tkzGetLength{AG}
		% draw
		\tkzDrawSegment(A,B)
		\tkzDrawPoints(A)
		\tkzCompass(A,G)
		\tkzCompass(G,H)
		\tkzCompass(A,I)
		%\begin{scope}[dim style/.style={dashed, sloped, teal}]
			%\tkzDrawSegment[dim={\pgfmathprintnumber\XY, -6pt, text=red}](X,Y)
			%\tkzDrawSegment[dim={\pgfmathprintnumber\XZ,  6pt, text=red}](X,Z)
			%\tkzDrawSegment[dim={\pgfmathprintnumber\YZ, -6pt, text=red}](Y,Z)  
			%\tkzDrawSegment[dim={\pgfmathprintnumber\AG, -6pt, text=red}](A,G)
			%\tkzDrawSegment[dim={\pgfmathprintnumber\AI, -6pt, text=red}](A,I)
			%\tkzDrawSegment[dim={\pgfmathprintnumber\HI, -6pt, text=red}](H,I)
			%\tkzDrawSegment[dim={\pgfmathprintnumber\AG, -6pt, text=red}](A,G)                  
		%\end{scope}
		% lab
		\tkzLabelPoints(A,G,H,I,B)
	\end{tikzpicture}
	\caption{}
\end{figure}    

\item Mark the intersect of $\mathscr{C}(A;AG)$ and $\mathscr{C}(H;HI)$, $K$. \hfill\textcolor{red}{I.22}

 \item Construct a triangle $\widetriangle{AKH}$ such that:
     
          \[\overline{AK} = \overline{ED}\]
           \[ \overline{AH} = \overline{CD}\]
           \[ \overline{KH} = \overline{CE}\]\hfill\textcolor{red}{I.22}
      

\begin{figure}[H]
		\begin{tikzpicture}
		% def
\tkzDefPoint(0,0){X}
\tkzDefPoint(2.5,-2){Y}
\tkzDefPoint(2,1.5){Z}
\tkzDefPoint(0,-5){A}
\tkzDefPoint(10,-5){B}
% calc
\tkzCalcLength(X,Y) \tkzGetLength{XY}
\tkzCalcLength(X,Z) \tkzGetLength{XZ}
\tkzCalcLength(Y,Z) \tkzGetLength{YZ}
\tkzDefPointWith[linear normed,K=\XY](A,B) \tkzGetPoint{G}
\tkzDefPointWith[linear normed,K=\YZ](G,B) \tkzGetPoint{H}
\tkzDefPointWith[linear normed,K=\XZ](A,B) \tkzGetPoint{I}
		% draw
		\tkzDrawSegment(A,B)
		\tkzDrawPoints(A,I,G,H)
		\tkzInterCC(G,H)(A,I)					\tkzGetPoints{J}{K}
		\tkzCalcLength(A,G) \tkzGetLength{AG}
		\tkzCalcLength(A,K) \tkzGetLength{AK}
		\tkzCalcLength(K,G) \tkzGetLength{KG}
		% draw
		\tkzDrawSegment(A,B)
		\tkzDrawPoints(A,G,H,I)
		\tkzCompass(G,K)
		\tkzCompass(H,K)
		%\begin{scope}[dim style/.style={dashed, sloped, teal}]
			%\tkzDrawSegment[red,dim={\pgfmathprintnumber\AG, -6pt, text=red}](A,G)
			%\tkzDrawSegment[red,dim={\pgfmathprintnumber\AK,  6pt, text=red}](A,K)
			%\tkzDrawSegment[red,dim={\pgfmathprintnumber\KG, -6pt, text=red}](K,G)                    
		%\end{scope}
		\tkzDrawSegments[red](A,K A,G K,G)
		% lab
		\tkzLabelPoints(A,B,G,H,I,K)
	\end{tikzpicture}
	\caption{}
\end{figure}    
    
\end{enumerate}

\begin{proof}

\begin{itemize}

\item[]

\item By construction, 
\[\angle{DCE} = \angle{KAG}\]

\item[$\therefore$] a rectilinear $\angle{KAG}$ has been constructed on the given straight line $\overline{AB}$ and at the point $A$ on it, equal to the given rectilinear $\angle{DCE}$. \hfill\textcolor{red}{I.8}

\end{itemize}

\end{proof}

\clearpage
