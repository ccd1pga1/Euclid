
\section*{Poroposition 43}

\begin{thm}
In any parallelogram, the complements of the parallelograms about the diameter equal one another.
\end{thm}

\begin{proof}

\begin{itemize}

\item[]

\begin{figure}[H]
	\begin{tikzpicture}
		\tkzDefPoint(1,4){A}
		\tkzDefPoint(0,0){B}
		\tkzDefPoint(4,0){C}
		\tkzDefPoint(5,4){D}
		\tkzDrawPolygon(A,B,C,D)
		\tkzDrawSegment(A,C)
		\tkzDrawPoints(A,B,C,D)
		\tkzLabelPoints[above](A,D)
		\tkzLabelPoints[below](B,C)
	\end{tikzpicture}
	\caption{}
\end{figure}

\item In parallelogram $AEKH$, since $\overline{AK}$ is its diameter:
\[\widetriangle{AEK} = \widetriangle{AHK}\] 
Similarly,  
\[\widetriangle{KFC} = \widetriangle{KGC}\]\hfill\textcolor{red}{I.34,C.N.2}

\begin{figure}[H]
	\begin{tikzpicture}
		\tkzDefPoint(1,4){A}
		\tkzDefPoint(0,0){B}	
		\tkzDefPoint(4,0){C}
		\tkzDefPoint(5,4){D}
		\tkzDefPoint(2,4){H}
		\tkzDefPoint(1,0){G}
		\tkzDrawPolygon(A,B,C,D)
		\tkzInterLL(A,C)(G,H) 	\tkzGetPoint{K}
		\tkzDefShiftPoint[K](0:-1){E}
		\tkzDefShiftPoint[K](0:3){F}
		\tkzLabelPoints[right](F)
		\tkzDrawSegments(A,C H,G E,F)
		\tkzDrawPolygon[fill=blue,  opacity=0.5](E,B,G,K)
		\tkzDrawPolygon[fill=green, opacity=0.7](H,K,F,D)
		\tkzDrawPoints(A,B,C,D,E,K)
		\tkzLabelPoints[above](A,D,H)
		\tkzLabelPoints[below](B,C,G)
		\tkzLabelPoints[left](E)
		\tkzLabelPoints[below left](K)
	\end{tikzpicture}
	\caption{}
\end{figure}

\item Since 
\[\widetriangle{AEK} = \widetriangle{AHK}\] \hfill\textcolor{red}{I.34}
and 
\[\widetriangle{KFC} = \widetriangle{KGC}\] \hfill\textcolor{red}{C.N.2}

\item[$\therefore$] 
\[\widetriangle{AEK} + \widetriangle{KGC} =  \widetriangle{AHK} + \widetriangle{KFC}\]\hfill\textcolor{red}{C.N.3}

\item Since the whole 
\[\widetriangle{ABC} = \widetriangle{ADC}\]

\item[$\therefore$]the remaining complement $BGKE$ equals the remaining complement $KFDH$.

\end{itemize}

\end{proof}

\clearpage
