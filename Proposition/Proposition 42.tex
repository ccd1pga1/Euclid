
\section*{Proposition 42}

\renewcommand\qedsymbol{Q.E.F}

\begin{con}
To construct a parallelogram equal to a given triangle in a given rectilinear angle.
\end{con}

Let $\widetriangle{ABC}$ be the given triangle, and $D$ the given rectilinear angle.

\begin{figure}[H]
	\begin{subfigure}{0.25\textwidth}
		\begin{tikzpicture}[scale=1.5]
			\tkzDefPoint(0,0){D}
			\tkzDefShiftPoint[D](0:1.5){Y}
			\tkzDefShiftPoint[D](40:1.5){Z}
			\tkzDrawPoints(D,Y,Z)
			\tkzDrawSegments(D,Y D,Z)
			\tkzLabelPoints(D)
		\end{tikzpicture}
		\caption{}
	\end{subfigure}
%%%%%%%%%%%%%%%%
	\begin{subfigure}{0.25\textwidth}
		\begin{tikzpicture}
			\TRI{0,0}{B}{0:3}{C}{40:3}{A}{green, opacity=0.3}
		\end{tikzpicture}
		\caption{}
	\end{subfigure}
	\caption{}
\end{figure}

\textbf{Construction:}

\begin{enumerate}
\item[]
    \item Bisect $\overline{BC}$ at $E$, and join $\overline{AE}$.
    \item Construct the angle $\angle{CEF}$ on $\overline{EC}$ at the point $E$ on it equal to the angle $D$.
    \item Draw $\overline{AG}$ through $A$ parallel to $\overline{EC}$.
    \item Draw $\overline{CG}$ through $C$ parallel to $\overline{EF}$.\hfill\textcolor{red}{I.10,I.Post.1,I.23,I.31}
\end{enumerate}

\begin{figure}[H]
	\begin{subfigure}{0.4\textwidth}
		\begin{tikzpicture}
			\TRI{0,0}{B}{0:3}{C}{40:3}{A}{green, opacity=0.3}
			\tkzDefPoint(1.5,0){E}
			\tkzDrawSegment(A,E)
			\tkzLabelPoints(E)
		\end{tikzpicture}
		\caption{}
	\end{subfigure}
%%%%%%%%%%%%%%
	\begin{subfigure}{0.4\textwidth}
		\begin{tikzpicture}
			\TRI{0,0}{B}{0:3}{C}{40:3}{A}{green, opacity=0.3}
			\tkzDefPoint(1.5,0){E}
			\tkzDrawSegment(A,E)
			\TRI{1.5,0}{E}{0:1.5}{C}{40:3}{F}{none}
		\end{tikzpicture}
		\caption{}
	\end{subfigure}
%%%%%%%%%%%%%%%%
	\begin{subfigure}{0.4\textwidth}
		\begin{tikzpicture}
			\TRI{0,0}{B}{0:3}{C}{40:3}{A}{green, opacity=0.3}
			\tkzDefPoint(1.5,0){E}
			\TRI{1.5,0}{E}{0:1.5}{C}{40:3}{F}{none}
			\tkzDefShiftPoint[A](0:3){G}
			\tkzDrawSegments(A,E A,G C,G)
			\tkzLabelPoints(G)
		\end{tikzpicture}
		\caption{}
	\end{subfigure}
	\caption{}
\end{figure}

\begin{proof}

\begin{itemize}

\item Parallelogram $FECG$ is a parallelogram. \hfill\textcolor{red}{I.38}

\item Since 
\[\overline{BE} = \overline{EC}\] 
\[\triangle{ABE} = \triangle{AEC} \]
as they are on equal bases $\overline{BE}$ and $\overline{EC}$ and in the same parallels $\overline{BC}$ and $\overline{AG}$.%\hfill\textcolor{red}{I.41}

\item The parallelogram $FECG$ is double $\widetriangle{AEC}$ since it has the same base and is in the same parallels with it. \hfill\textcolor{red}{C.N.1}

The parallelogram $FECG$ has been constructed equal to the given $\widetriangle{ABC}$ in the $\angle{CEF}$ which equals $D$.

\end{itemize}

\end{proof}

\clearpage
