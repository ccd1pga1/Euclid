
\section*{Proposition 1}

\renewcommand\qedsymbol{Q.E.F}
    
\begin{con}
        To construct an equilateral triangle on a given finite straight line $\overline{AB}$.
\end{con}

Let $\overline{AB}$ be the given finite straight line.\hfill\textit{ I.1}

\begin{figure}[H]
	\begin{tikzpicture}
			\tkzDefPoint(0,0){A}
			\tkzDefPoint(3,0){B}
			\tkzDrawPoints(A,B)
			\tkzLabelPoints(A,B)
			\tkzDrawSegment[red](A,B)
	\end{tikzpicture}
		\caption{Equilateral triangle construction}
\end{figure}


\begin{enumerate}
  	\item  It is required to construct an equilateral triangle on the straight line $\overline{AB}$.  
  	\item Describe the $\mathscr{C}(A;AB)$.  
  	\item Then,  describe the $\mathscr{C}(A;BA)$. 
  	\item Join the straight lines $\overline{CA}$ and $\overline{CB}$ from the point $C$ at which the circles cut one another to the points $A$ and $B$. \hfill\textit{ I.Post.3, I.Post.1}
  \end{enumerate}

  
\begin{figure}[H]
\centering
	\begin{subfigure}{0.3\textwidth}
		\begin{tikzpicture}
		\centering
    			\tkzDefPoint(0,0){A}
    			\tkzDefPoint(1.5,0){B}
    			\tkzDrawCircle[blue](A,B)
    			\tkzDrawPoints(A,B)
    			\tkzLabelPoints(A,B)
    			\tkzDrawSegment[red](A,B)
		\end{tikzpicture}
		\sidecaption[a][-3cm]{Step 1}
	\end{subfigure}
%%%%%%%%%%%%%%%%%%%%%%%%%%%%
	\begin{subfigure}{0.3\textwidth}
		\begin{tikzpicture}
		\centering
   			\tkzDefPoint(0,0){A}
    			\tkzDefPoint(1.5,0){B}
    			\tkzDrawCircle[green](B,A)
    			\tkzInterCC(A,B)(B,A)       \tkzGetPoints{C}{D}
    			\tkzCompass[color=blue, thick](A,C)
    			\tkzDrawPoints(A,B)
    			\tkzLabelPoints(A,B)
    			\tkzDrawSegment[red](A,B)
		\end{tikzpicture}
			\sidecaption[b][-2cm]{Step 2}
	\end{subfigure}
%%%%%%%%%%%%%%%%%%%%%%%%%%%%
	\begin{subfigure}{0.3\textwidth}
		\begin{tikzpicture}
		\centering
			\tkzDefPoint(0,0){A}
    			\tkzDefPoint(1.5,0){B}
			\tkzInterCC(A,B)(B,A)		\tkzGetPoints{C}{D}
			\tkzCompass[color=blue, thick](A,C)
			\tkzCompass[color=green, thick](B,C)s
			\tkzDrawPoints(A,B,C)
       			\tkzLabelPoints(A,B)
       			\tkzLabelPoints[above](C)
         		\tkzDrawSegments[red](A,B A,C B,C)
		\end{tikzpicture}
		\sidecaption[c][-1cm]{Step 3}
	\end{subfigure}
	\caption{Equilateral triangle construction}
\end{figure}

\begin{proof}
 
  \begin{itemize}
  \item Since the point $A$ is the center of the $\mathscr{C}(A;AB)$, $\therefore$ $overline{AC}$ equals $\overline{AB}$. Again, since the point $B$ is the center of the $\mathscr{C}(A;BA)$ $\therefore$ $\overline{BC}$ equals $\overline{BA}$.  \hfill\textit{ I.Def.15}
  \item Things which equal the same thing also equal one another $\therefore$ $\overline{AC}$ also equals$\overline{BC}$. 
              \hfill\textit{ C.N.1}
\end{itemize}

\clearpage

\textbf{Conclusion:}

\begin{itemize}
	\item $\therefore$,  the three straight lines $\overline{AC}$, $\overline{AB}$, and $\overline{BC}$ equal one another.
  	\item $\therefore$,  $\widetriangle{ABC}$ is equilateral,  and it has been constructed on the given finite straight line $\overline{AB}$. \hfill\textit{ I.Def.20}
\end{itemize}

\begin{figure}[h]
	\begin{tikzpicture}
		\tkzDefPoint(0,0){A}
		\tkzDefPoint(3,0){B}
		\tkzInterCC(A,B)(B,A)		\tkzGetPoints{C}{D}
		\tkzCompass[color=blue, thick](A,C)
		\tkzCompass[color=green, thick](B,C)
		\tkzDrawPoints(A,B,C)
       		\tkzLabelPoints(A,B)
       		\tkzLabelPoints[above](C)
         	\tkzDrawSegments[red](A,B A,C B,C)
	\end{tikzpicture}
		\caption{Equilateral triangle construction}
\end{figure}

 The construction has been successfully completed. 
\end{proof}

%\begin{remark}
	\begin{itemize}
		\item The construction relies on the principles of circle description and intersection, line segment joining, and the properties of circles with given centers. \hfill\textcolor{red}{ I.Post.1, I.Post.3, I.Def.15}
    		\item The Equilateral Triangle is characterized by the equality of its three sides ($\overline{AC}$, $\overline{AB}$, $\overline{BC}$).\hfill\textit{ I.Def.20}
	\end{itemize}
%\end{remark}

\clearpage
