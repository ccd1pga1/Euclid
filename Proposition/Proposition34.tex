
\section*{Proposition 34}

\begin{thm}
In a parallelogram, opposite sides and angles are equal, and the diameter bisects the area.
\end{thm}

\begin{figure}[H]
	\begin{tikzpicture}
		\tkzDefPoints{0/0/C,3/0/D,2/2/B}
		\tkzDefParallelogram(C,D,B)	\tkzGetPoint{A}
		\tkzDrawPolygon[fill=blue,  opacity=0.3](C,D,B,A)
		\tkzDrawSegment(C,B)
		\tkzLabelPoints[right](D,B)
		\tkzLabelPoints[left](C,A)
		\tkzDrawPoints(A,...,D)
	\end{tikzpicture}
	\caption{}
\end{figure}

\textbf{Given:} Parallelogram $ACDB$ with diameter $BC$.

\begin{proof}

\textbf{1. Opposite Angles are Equal} \hfill\textcolor{red}{ I.34, I.29}
\begin{itemize}
   \item Since $\overline{AB} \parallel \overline{CD}$ and $\overline{BC}$ is a transversal line, we have  
   \[\angle{ABC} = \angle{BCD}\] \hfill\textcolor{red}{I.34}
   
    \item Similarly, since 
    $\overline{AC} \parallel \overline{BD}$ and $\overline{BC}$ is a transversal line,we have 
    \[\angle{ACB} = \angle{CBD}\] \hfill\textcolor{red}{I.29}
\end{itemize}

\textbf{2. Triangle Equality} \hfill\textcolor{red}{ I.26}
\begin{itemize}
    \item Consider $\widetriangle{ABC}$ and $\widetriangle{DCB}$ 
    
    \item They have 
    \[\angle{ABC} = \angle{BDC}\]
    \[ \angle{ACB} = \angle{CBD}\]
    and $\overline{BC}$ in common.
    
    \item[$\therefore$]by SAS criterion,  
    \[\widetriangle{ABC} \cong \widetriangle{DCB}\]
     
    \item This implies  
    \[\overline{AB} = \overline{CD}\]
     \[\overline{AC} = \overline{BD}\]
     \[\angle{BAC} = \angle{CDB}\]
\end{itemize}

\textbf{3. Sum of Angles} \hfill\textcolor{red}{C.N.2}
\begin{itemize}
    \item $\angle{ABC} = \angle{BCD}$ and $\angle{CBD} = \angle{ACB}$, 
    \item imply that the sum of angles $\angle{ABD} = \angle{ACD}$.
\end{itemize}

\textbf{4. Conclusion - Opposite Sides and Angles:}
\begin{itemize}
    \item In parallelogram $ACDB$, 
    \item opposite sides 
    \[\overline{AB} = \overline{CD}\] 
    and 
    \[\overline{AC} = \overline{BD}\]
     
    \item and opposite angles 
    \[\angle{ABC} = \angle{BDC}\] 
    and 
    \[\angle{ACB} = \angle{CBD}\] 
\end{itemize}

\clearpage

\textbf{5. Diameter Bisects the Parallelogram} \hfill\textcolor{red}{ I.4}
\begin{itemize}
    \item Since 
    \[\overline{AB} = \overline{CD}\] 
    $\overline{BC}$ is common, and 
    \[\angle{ABC} = \angle{BCD}\]
     
    \item 
    \[\widetriangle{ABC} \cong \widetriangle{BCD}\]
    
    \item This implies 
    \[\overline{AC} = \overline{DB}\]
    \[ \widetriangle{ABC} = \widetriangle{DCB}\]
\end{itemize}

\textbf{6. Conclusion - Diameter Bisects the Area:}
\begin{itemize}
    \item[$\therefore$] the diameter $\overline{BC}$ bisects the parallelogram $ACDB$.
\end{itemize}

\end{proof}

\clearpage
