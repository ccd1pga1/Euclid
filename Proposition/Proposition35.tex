
\section*{Proposition 35}

\begin{thm}
Parallelograms on the same base and in the same parallels are equal to one another.
\end{thm}

\begin{figure}[H]
	\begin{tikzpicture}
		\tkzDefPoint(0,0){B}
		\tkzDefPoint(2,0){C}
		\tkzDefPoint(-1,3){A}
		\tkzDefParallelogram(A,B,C)		 \tkzGetPoint{D}
		\tkzDefPointOnLine[pos=.60](C,D)	\tkzGetPoint{G}
		\tkzInterLL(A,D)(B,G) 			\tkzGetPoint{E}
		\tkzDefParallelogram(E,B,C) \tkzGetPoint{F}
		\tkzDrawSegment(D,E)
		\tkzDrawPolygons(A,B,C,D B,C,F,E)
		\tkzLabelPoints(B,C)
		\tkzLabelPoints[above](A,D,E,F)
		\tkzLabelPoints[right](G)
		\tkzDrawPoints(A,...,G)
	\end{tikzpicture}
	\caption{}
\end{figure}

\textbf{Given:} Parallelograms $ABCD$ and $EBCF$ on the same base $\overline{BC}$ and in the same parallels $\overline{AF}$ and $\overline{BC}$.

\begin{proof}

\begin{enumerate}
    \item \textbf{Parallelogram Properties} \hfill\textcolor{red}{I.35, I.34}
    \begin{itemize}
        \item Since $ABCD$ is a parallelogram, 
        \[\overline{AD} = \overline{BC}\] \hfill\textcolor{red}{I.35}
        
        \item Similarly, for $EBCF$, 
        \[\overline{EF} = \overline{BC}\] \hfill\textcolor{red}{I.34}
    \end{itemize}

    \item \textbf{Triangle Equality} \hfill\textcolor{red}{C.N.1, C.N.2, I.4}
    \begin{itemize}
        \item Since 
        \[\overline{AD} = \overline{BC}\] 
        and 
        \[\overline{EF} = \overline{BC}\] 
        we can conclude that 
        \[\overline{AD} = \overline{EF}\]
        
        \item Since $\overline{DE}$ is common, we have 
        \[\overline{AE} = \overline{DF}\]\hfill\textcolor{red}{C.N.1}
        
        \item Also, 
        \[\overline{AB} = \overline{DC}\]\hfill\textcolor{red}{I.34}
        
        \item[$\therefore$] the two sides $\overline{EA}$ and $\overline{AB}$ equal the two sides $\overline{FD}$ and $\overline{DC}$, respectively, and $\angle{FDC}$ = $\angle{EAB}$. \hfill\textcolor{red}{I.4}
        
        \item Thus, the base 
        \[\overline{EB} =\overline{FC}\] 
        and 
        \[\widetriangle{EAB} = \widetriangle{FDC}\]\hfill\textcolor{red}{C.N.2}
    \end{itemize}

    \item \textbf{Trapezium Equality} \hfill\textcolor{red}{C.N.3}
    \begin{itemize}
        \item Subtract $\widetriangle{DGE}$ from each side.
        \item Then the trapezium $ABGD$ equals the trapezium $EGCF$.
    \end{itemize}

    \item \textbf{Whole Parallelogram Equality. }\hfill\textcolor{red}{C.N.2}
    \begin{itemize}
        \item Add the $\widetriangle{GBC}$ to each side.
        \item Then the whole parallelogram $ABCD$ equals the whole parallelogram $EBCF$.
    \end{itemize}
\end{enumerate}

\clearpage

\textbf{Conclusion:}
\begin{itemize}
    \item[$\therefore$] parallelograms on the same base and in the same parallels equal one another.
\end{itemize}

\end{proof}

\clearpage
