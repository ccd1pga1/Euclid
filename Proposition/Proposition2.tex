\section*{Proposition 2}
\renewcommand{\qedsymbol}{Q.E.F}
  
\begin{con}
To place a straight line equal to a given straight line with one end at a given point $A$. 
\end{con}

Let $\overline{BC}$ be the given straight line.
   
	\textbf{Con}
\begin{enumerate}
	\item   Given a point $A$ and the given straight finite line$\overline{BC}$. 

\begin{figure}[ht]    
\centering
	\begin{tikzpicture}
		\tkzDefPoint(0,0){A}
		\tkzDefPoint(1.5,0.5){B}
		\tkzDefPoint(2,3){C}
		\tkzDrawSegment[red](B,C)
		\tkzDrawPoints[fill=gray](A,B,C)
		\tkzLabelPoints[left](A)
		\tkzLabelPoints[right,red](B)
		\tkzLabelPoints[above,red](C)
	\end{tikzpicture}
	\caption{step1}
\end{figure} 	

	\item Join the straight line $\overline{AB}$ from point $A$ to point $B$, and construct the equilateral $\widetriangle{DAB}$ on it. \hfill\textcolor{red}{ Post.1, I.1}

\begin{figure}[H]
\centering
	\begin{tikzpicture}[scale=1.5]
		\tkzDefPoint(0,0){A}		
		\tkzDefPoint(1.5,0.5){B}
		\tkzDefPoint(2,3){C}
		\tkzDefTriangle[equilateral](A,B) 	\tkzGetPoint{D}
		\tkzDrawPolygon[blue](A,B,D)
		\tkzLabelPoints[left](A)  
		\tkzLabelPoints[above](D)
		\tkzLabelPoints[right,red](B)
		\tkzLabelPoints[above,red](C)
		\tkzCompass(A,D)
		\tkzCompass(B,D)
		\tkzDrawSegment[red](B,C)
		\tkzDrawPoints[fill=gray](A,D)
		\tkzDrawPoints[red](B,C)
	\end{tikzpicture}
 	\caption{step 2}
\end{figure}

\item Produce the straight lines $\overline{AE}$ and $\overline{BF}$ in a straight line with $\overline{DA}$ and $\overline{DB}$. 
\item Describe  $\mathscr{C}(B;BC)$, getting point $G$, the intersect of this circle with $\overline{BF}$ . 

\begin{figure}[H]
\centering
	\begin{subfigure}{0.4\textwidth}
		\begin{tikzpicture}[scale=0.8]
			\tkzDefPoint(0,0){A}		
		\tkzDefPoint(1.5,0.5){B}
		\tkzDefPoint(2,3){C}
			\tkzDefTriangle[equilateral](A,B) 		\tkzGetPoint{D}
			\tkzDefPointOnLine[pos=3](D,A)		\tkzGetPoint{E} 
			\tkzDefPointOnLine[pos=3](D,B)		\tkzGetPoint{F}
			\tkzDrawSegments(A,E B,F)
			\tkzDrawSegment[red](B,C)
			\tkzDrawPolygon[blue](A,B,D)
			\tkzLabelPoints[left](A)  
			\tkzLabelPoints[above](D)
			\tkzLabelPoints[right,red](B)
			\tkzLabelPoints[above,red](C)
			\tkzLabelPoints(E,F)
			\tkzDrawPoints[red](B,C)
			\tkzDrawPoints[fill=gray](A,B,C,D,E,F)
		\end{tikzpicture}
		\caption{Produce the line}
	\end{subfigure}
%%%%%%%%%%%%%%%%%%%%%%%%%%%%%%
	\begin{subfigure}{0.4\textwidth}
		\begin{tikzpicture}[scale=0.8]
			\tkzDefPoint(0,0){A}		
			\tkzDefPoint(1.5,0.5){B}
			\tkzDefPoint(2,3){C}
			\tkzDefTriangle[equilateral](A,B) 		\tkzGetPoint{D}
			\tkzDefPointOnLine[pos=3](D,A)		\tkzGetPoint{E} 
			\tkzDefPointOnLine[pos=3](D,B)		\tkzGetPoint{F}
			\tkzDrawSegments(A,E B,F)
			\tkzDrawSegment[red](B,C)
			\tkzInterLC(B,F)(B,C)				\tkzGetPoints{H}{G}
			\tkzDrawPolygon[blue](A,B,D)
			\tkzDrawCircle[green](B,C)
			\tkzLabelPoints[left](A)  
			\tkzLabelPoints[above](D)
			\tkzLabelPoints[right,red](B)
			\tkzLabelPoints[above,red](C)
			\tkzLabelPoints[right](G)
			\tkzLabelPoints(E,F)
			\tkzDrawPoints[red](B,C)
			\tkzDrawPoints[fill=gray](A,D,E,F,G)
		\end{tikzpicture}
		\caption{draw a circle}
	\end{subfigure}
%%%%%%%%%%%%%%%%%%%%%%%%%%%%%%

\item Now, describe $\mathscr{C}(D;DG)$, the intersect of this circle and $\overline{AE}$ gives point $L$.  

%%%%%%%%%%%%%%%%%%%%%%%%%%%%%%
	\begin{subfigure}{0.55\textwidth}
		\begin{tikzpicture}[scale=0.7]
			\tkzDefPoint(0,0){A}		
			\tkzDefPoint(1.5,0.5){B}
			\tkzDefPoint(2,3){C}
			\tkzDefTriangle[equilateral](A,B) 		\tkzGetPoint{D}
			\tkzDefPointOnLine[pos=3](D,A)		\tkzGetPoint{E} 
			\tkzDefPointOnLine[pos=3](D,B)		\tkzGetPoint{F}
			\tkzDrawSegments(A,E B,F)
			\tkzDrawSegment[red](B,C)
			\tkzInterLC(B,F)(B,C)				\tkzGetPoints{H}{G}
			\tkzInterLC(D,E)(D,G)				\tkzGetPoints{K}{L}
			\tkzDrawPolygon[blue](A,B,D)
			\tkzDrawCircle[green](B,C)
			\tkzDrawCircle[purple](D,G)
			\tkzLabelPoints[left](A,L)  
			\tkzLabelPoints[above](D)
			\tkzLabelPoints[right,red](B)
			\tkzLabelPoints[above,red](C)
			\tkzLabelPoints[right](G)
			\tkzLabelPoints(E,F)
			\tkzDrawPoints[red](B,C)
			\tkzDrawPoints[fill=gray](A,D,E,F,G,L)
		\end{tikzpicture}	
		\caption{intersection}
	\end{subfigure}
	\caption{something}
\end{figure}
\hfill\textcolor{red}{ Post.2,Post.3}

\clearpage

\end{enumerate}

\begin{proof}

\begin{itemize}

\item Since the point $B$ is the center of the circle with radius $BC$, $\therefore$ $\overline{BC}$ equals $\overline{BG}$.  Again, since the point $D$ is the center of the circle with radius $DG$, $\therefore$ $\overline{DL}$ equals $\overline{DG}$. \hfill\textcolor{red}{ I.Def.15}

\begin{figure}[H]
\centering
	\begin{tikzpicture}[scale=0.7]
		\tkzDefPoint(0,0){A}		
		\tkzDefPoint(1.5,0.5){B}
		\tkzDefPoint(2,3){C}
		\tkzDefTriangle[equilateral](A,B) 		\tkzGetPoint{D}
		\tkzDefPointOnLine[pos=3](D,A)		\tkzGetPoint{E} 
		\tkzDefPointOnLine[pos=3](D,B)		\tkzGetPoint{F}
		\tkzDrawSegments(L,E G,F)
		\tkzDrawSegments[red, thick](B,C A,L B,G)
		\tkzInterLC(B,F)(B,C)				\tkzGetPoints{H}{G}
		\tkzInterLC(D,E)(D,G)				\tkzGetPoints{K}{L}
		\tkzDrawPolygon[blue](A,B,D)
		\tkzDrawCircle[green](B,C)
		\tkzDrawCircle[purple](D,G)
		\tkzLabelPoints[left](A,L)  
		\tkzLabelPoints[above](D)
		\tkzLabelPoints[right,red](B)
		\tkzLabelPoints[above,red](C)
		\tkzLabelPoints[right](G)
		\tkzLabelPoints(E,F)
		\tkzDrawPoints[red](B,C)
		\tkzDrawPoints[fill=gray](A,D,E,F,G,L)
	\end{tikzpicture}
	\caption{end}
\end{figure}

	\item In these,  $\overline{DA}$ equals $\overline{DB}$, $\therefore$ the remainder $AL$ equals the remainder $BG$. \hfill\textcolor{red}{C.N.3}
	\item But $\overline{BC}$ was also proved equal to $\overline{BG}$, $\therefore$ each of the straight lines $\overline{AL}$ and $\overline{BC}$ equals $\overline{BG}$. And things which equal the same thing also equal one another $\therefore$ $\overline{AL}$ also equals $\overline{BC}$.

\end{itemize}
																	
\textbf{Conclusion:}
\begin{itemize}
	\item $\therefore$ the straight line $\overline{AL}$ equal to the given straight line $\overline{BC}$ has been placed with one end at the given point $A$.
	\item The construction has been successfully completed. 
\end{itemize}

\end{proof}

\clearpage
