\section*{Proposition 5}

\begin{thm}
In isosceles triangles, the angles at the base equal one another, and if the equal straight lines are produced further, then the angles under the base equal one another.
\end{thm}

\begin{figure}[H]
\centering
	\begin{tikzpicture}
		\Tri{B}{3}{0}{C}{1.5}{3}{A}{none}
		\tkzDrawPolygon(A,B,C)
		\tkzMarkSegment[mark=|](A,C)
		\tkzMarkSegment[mark=|](A,B)
		\tkzMarkAngle[size=0.5](A,C,B)
		\tkzMarkAngle[size=0.5](C,B,A)
	\end{tikzpicture}
	\caption{Isosceles Triangle Theorem}
\end{figure}

Let $\widetriangle{ABC}$ be an isosceles triangle with side $\overline{AB}$ equal to side $\overline{AC}$, and let the straight lines $\overline{BD}$ and $\overline{CE}$ be produced further in a straight line with $\overline{AB}$ and $\overline{AC}$.\hfill \textcolor{red}{ I.Def.20, I.Post.2}

\begin{figure}[H]
\centering
	\begin{tikzpicture}
		\Tri{B}{3}{0}{C}{1.5}{3}{A}{none}
       		\tkzDefPointOnLine[pos=2](A,B)		\tkzGetPoint{E}
        		\tkzDefPointOnLine[pos=2](A,C)		\tkzGetPoint{D}
        		\tkzDrawPolygon(A,B,C)
       		\tkzDrawSegment(A,E)
        		\tkzDrawSegment(A,D)
         	\tkzLabelPoints[left](E)
         	\tkzLabelPoints[right](D)
         	\tkzMarkSegment[mark=|](A,C)
        		\tkzMarkSegment[mark=|](A,B)
    		\tkzMarkAngle[size=0.5](A,C,B)
		\tkzMarkAngle[size=0.5](C,B,A)
	\end{tikzpicture}
	\caption{Isosceles Triangle Theorem; step 1}
\end{figure}

\subsection*{Definitions and Postulates:}
\begin{itemize}
  	\item An isosceles triangle is a triangle with two sides of equal length. \hfill\textcolor{red}{ I.Def.20}
  	\item A straight line can be produced indefinitely.\hfill\textcolor{red}{ I.Post.2}
  	\item Given two points, a straight line can be drawn between them.\hfill\textcolor{red}{ I.Post.1}
\end{itemize}

\begin{lemma}
If two sides and the included angle of one triangle are equal to the corresponding sides and angle of another triangle, then the two triangles are congruent.\hfill\textcolor{red}{ I.3}
\end{lemma}

\begin{figure}[H]
\centering
	\begin{tikzpicture}
		\Tri{B}{1.5}{0}{C}{0.75}{2}{A}{none}
     		\tkzDefPointOnLine[pos=2](A,B)		\tkzGetPoint{E}
     		\tkzDefPointOnLine[pos=2](A,C)		\tkzGetPoint{D}
     		\tkzDefPointOnLine[pos=1.5](A,B)		\tkzGetPoint{F}
     		\tkzDefPointOnLine[pos=1.5](A,C)		\tkzGetPoint{G} 
     		\tkzDrawPolygon(A,B,C)
     		\tkzDrawSegment(A,E)
     		\tkzDrawSegment(A,D)
     		\tkzLabelPoints[left](E)
     		\tkzLabelPoints[right](D)
     		\tkzLabelPoints[left](F)
     		\tkzLabelPoints[right](G)
     		\tkzMarkSegment[mark=|](A,C)
     		\tkzMarkSegment[mark=|](A,B)
     		\tkzMarkAngle[size=0.5](A,C,B)
     		\tkzMarkAngle[size=0.5](C,B,A)
     		\tkzDrawPoints[fill=gray](A,B,C,D,E,F,G)
	\end{tikzpicture}
	\caption{Isosceles Triangle Theorem; step 2}
\end{figure}
    
\clearpage
    
\begin{con}
	\begin{enumerate}
	\item[]
    		\item Take an arbitrary point $F$ on $\overline{BD}$.
    		\item Cut off $\overline{AG}$ from $\overline{AE}$,  where $\overline{AG} > \overline{AF}$.
    		\item Join $FC$ and $GB$.\hfill\textcolor{red}{ I.3, I.Post.1}
	\end{enumerate}
\end{con}

\begin{figure}[H]
\centering
	\begin{subfigure}{0.35\textwidth}
		\begin{tikzpicture}
			\Tri{B}{1.5}{0}{C}{0.75}{2}{A}{none}
        			\tkzDefPointOnLine[pos=2](A,B)		\tkzGetPoint{E}
     			\tkzDefPointOnLine[pos=2](A,C)		\tkzGetPoint{D}
     			\tkzDefPointOnLine[pos=1.5](A,B)		\tkzGetPoint{F}
     			\tkzDefPointOnLine[pos=1.5](A,C)		\tkzGetPoint{G} 
         		\tkzDrawPolygon(A,B,C)
         		\tkzDrawSegment(A,E)
         		\tkzDrawSegment(A,D)
        			\tkzLabelPoints[left](E)
        			\tkzLabelPoints[right](D)
        			\tkzLabelPoints[left](F)
        			\tkzLabelPoints[right](G)
        			\tkzMarkSegment[mark=|](A,C)
        			\tkzMarkSegment[mark=|](A,B)
        			\tkzMarkAngle[size=0.5](A,C,B)
        			\tkzMarkAngle[size=0.5](C,B,A)
        			\tkzDrawPoints[fill=gray](A,B,C,D,E,F,G)
         		\tkzDrawSegment[dashed](C,F)
		\end{tikzpicture}
		\sidecaption[a][-4cm]{Join $\overline{FC}$}
	\end{subfigure}
%%%%%%%%%%%%
	\begin{subfigure}{0.35\textwidth}
		\begin{tikzpicture}
       			\Tri{B}{1.5}{0}{C}{0.75}{2}{A}{none}
       			\tkzDefPointOnLine[pos=2](A,B)		\tkzGetPoint{E}
 			\tkzDefPointOnLine[pos=2](A,C)		\tkzGetPoint{D}
     			\tkzDefPointOnLine[pos=1.5](A,B)		\tkzGetPoint{F}
     			\tkzDefPointOnLine[pos=1.5](A,C)		\tkzGetPoint{G} 
        			\tkzDrawPolygon(A,B,C)
        			\tkzDrawSegment(A,E)
        			\tkzDrawSegment(A,D)
        			\tkzLabelPoints[left](E)
        			\tkzLabelPoints[right](D)
        			\tkzLabelPoints[left](F)
        			\tkzLabelPoints[right](G)       
        			\tkzMarkSegment[mark=|](A,C)
        			\tkzMarkSegment[mark=|](A,B)
        			\tkzMarkAngle[size=0.5](A,C,B)
        			\tkzMarkAngle[size=0.5](C,B,A)
        			\tkzDrawSegment[dashed](B,G)
        			\tkzDrawPoints[fill=gray](A,B,C,D,E,F,G)
		\end{tikzpicture}
		\sidecaption[b][-3cm]{Join $\overline{GB}$}
	\end{subfigure}
%%%%%%%%%%%%%
	\begin{subfigure}{0.35\textwidth}
   		 \begin{tikzpicture}
        			\Tri{B}{1.5}{0}{C}{0.75}{2}{A}{none}
          		\tkzDefPointOnLine[pos=2](A,B)		\tkzGetPoint{E}
     			\tkzDefPointOnLine[pos=2](A,C)		\tkzGetPoint{D}
     			\tkzDefPointOnLine[pos=1.5](A,B)		\tkzGetPoint{F}
     			\tkzDefPointOnLine[pos=1.5](A,C)		\tkzGetPoint{G} 
        			\tkzDrawPolygon(A,B,C)
        			\tkzDrawSegment(A,E)
        			\tkzDrawSegment(A,D)
        			\tkzLabelPoints[left](E)
        			\tkzLabelPoints[right](D)
        			\tkzLabelPoints[left](F)
        			\tkzLabelPoints[right](G)       
        			\tkzMarkSegment[mark=|](A,C)
        			\tkzMarkSegment[mark=|](A,B)
        			\tkzMarkAngle[size=0.5](A,C,B)
        			\tkzMarkAngle[size=0.5](C,B,A)
        			\tkzDrawSegment[dashed](B,G)
          		\tkzDrawSegment[dashed](C,F)
        			\tkzDrawPoints[fill=gray](A,B,C,D,E,F,G)
		\end{tikzpicture}
		\sidecaption[c][-2cm]{$\overline{FC}$ $\overline{GB}$}
	\end{subfigure}
	\caption{Isosceles Triangle Theorem: step 3}
\end{figure}

\clearpage

\begin{proof}

\begin{itemize}
\item[]
	\item Since $\overline{AF} = \overline{AG}$ and $\overline{AB} = \overline{AC}$,  $\widetriangle{FAG}$ and $\widetriangle{ CAB}$ are congruent by $SSS$ (side-side-side) criterion.
        		\[\Rightarrow \angle{FAG} = \angle{CAB}\]
	\item $FC = GB$ (Base angles in congruent triangles are equal).\hfill\textcolor{red}{ I.4}
        		\[\Rightarrow \widetriangle AFC \cong \widetriangle AGB\]
        		\[\Rightarrow \angle{ACF} = \angle{ABG}, \quad \angle{AFC} = \angle{AGB}\]
	\item $BF = CG$. \hfill\textcolor{red}{ C.N.3}
        		\[\Rightarrow BF + FC = CG + GB\]
        		\[\Rightarrow \widetriangle BFC \cong \widetriangle CGB\]
        		\[\Rightarrow \angle{BFC} = \angle{CGB}, \quad \angle{BCF} = \angle{CBG}\]
	\item Combining the results:
        		\[\Rightarrow \angle{ACF} = \angle{ABG}\]
        		\[\Rightarrow \angle{AFC} = \angle{AGB}\]
        		\[\Rightarrow \angle{BFC} = \angle{CGB}\]
        		\[\Rightarrow \angle{BCF} = \angle{CBG}\]
	\item Since $\angle{ACF} = \angle{ABG}$ and $\angle{CBG} = \angle{BCF}$,  $BC$ is parallel to $FG$. \hfill\textcolor{red}{ I.Post.1}
	
\clearpage
	
	\item Since $\angle{ABC}$ and $\angle{ACB}$ are corresponding angles when $BC$ is parallel to $FG$, $\angle{ABC} = \angle{ACB}$.
	
	\item Also, $\angle{FBC} = \angle{GCB}$ as proved earlier.
\end{itemize}

\begin{figure}[H]
\centering
	\begin{subfigure}{0.4\textwidth}
		\begin{tikzpicture}
	 		\Tri{B}{1.5}{0}{C}{0.75}{2}{A}{none}
          		\tkzDefPointOnLine[pos=2](A,B)		\tkzGetPoint{E}
     			\tkzDefPointOnLine[pos=2](A,C)		\tkzGetPoint{D}
     			\tkzDefPointOnLine[pos=1.5](A,B)		\tkzGetPoint{F}
     			\tkzDefPointOnLine[pos=1.5](A,C)		\tkzGetPoint{G} 
        			\tkzDrawPolygon(A,B,C)
            		\tkzDrawPolygon[fill=green,opacity=0.5](F,B,C)
      			\tkzDrawSegment(A,E)
        			\tkzDrawSegment(A,D)
        			\tkzLabelPoints[left](E)
        			\tkzLabelPoints[right](D)
        			\tkzLabelPoints[left](F)
       			\tkzLabelPoints[right](G)       
        			\tkzMarkSegment[mark=|](A,C)
        			\tkzMarkSegment[mark=|](A,B)
        			\tkzMarkAngle[size=0.5](A,C,B)
        			\tkzMarkAngle[size=0.5](C,B,A)
        			\tkzDrawSegment[dashed](B,G)
         		\tkzDrawSegment[dashed](C,F)
        			\tkzDrawPoints[fill=gray](A,B,C,D,E,F,G)
    		\end{tikzpicture}
    		\caption{}
	\end{subfigure}
%%%%%%%%%%%%%
	\begin{subfigure}{0.4\textwidth}
		\begin{tikzpicture}
	 		\Tri{B}{1.5}{0}{C}{0.75}{2}{A}{none}
          		\tkzDefPointOnLine[pos=2](A,B)		\tkzGetPoint{E}
     			\tkzDefPointOnLine[pos=2](A,C)		\tkzGetPoint{D}
     			\tkzDefPointOnLine[pos=1.5](A,B)		\tkzGetPoint{F}
     			\tkzDefPointOnLine[pos=1.5](A,C)		\tkzGetPoint{G} 
         		\tkzDrawPolygon(A,B,C)
         		\tkzDrawPolygon[fill=red,opacity=0.5](B,C,G)
        			\tkzDrawSegment(A,E)
       			\tkzDrawSegment(A,D)
        			\tkzLabelPoints[left](E)
        			\tkzLabelPoints[right](D)
        			\tkzLabelPoints[left](F)
        			\tkzLabelPoints[right](G)       
        			\tkzMarkSegment[mark=|](A,C)
        			\tkzMarkSegment[mark=|](A,B)
        			\tkzMarkAngle[size=0.5](A,C,B)
        			\tkzMarkAngle[size=0.5](C,B,A)
        			\tkzDrawSegment[dashed](B,G)
         		\tkzDrawSegment[dashed](C,F)
        			\tkzDrawPoints[fill=gray](A,B,C,D,E,F,G)
		\end{tikzpicture}
		\caption{}
	\end{subfigure}
	\caption{Isosceles Triangle Theorem: step 4}
\end{figure}

$\therefore$ in isosceles $\widetriangle{ABC}$, the angles at the base ($\angle{ABC}$ and $\angle{ACB}$) are equal, and if the equal sides are produced further, the angles under the base ($\angle{FBC}$ and $\angle{GCB}$) are equal.

\begin{figure}[H]
\centering
	\begin{tikzpicture}
  		\Tri{B}{1.5}{0}{C}{0.75}{2}{A}{none}
          	\tkzDefPointOnLine[pos=2](A,B)		\tkzGetPoint{E}
     		\tkzDefPointOnLine[pos=2](A,C)		\tkzGetPoint{D}
     		\tkzDefPointOnLine[pos=1.5](A,B)		\tkzGetPoint{F}
     		\tkzDefPointOnLine[pos=1.5](A,C)		\tkzGetPoint{G} 
         	\tkzDrawPolygon(A,B,C)
        		\tkzDrawSegment(A,E)
        		\tkzDrawSegment(A,D)
        		\tkzLabelPoints[left](E)
        		\tkzLabelPoints[right](D)
       		\tkzLabelPoints[left](F)
        		\tkzLabelPoints[right](G)       
        		\tkzMarkSegment[mark=|](A,C)
        		\tkzMarkSegment[mark=|](A,B)
        		\tkzMarkAngle[size=0.5,arc=l,red](A,C,B)
       		\tkzMarkAngle[size=0.5,arc=l,red](C,B,A)
         	\tkzMarkAngle[size=0.5,arc=ll,blue](F,B,C)
        		\tkzMarkAngle[size=0.5,arc=ll,blue](B,C,G)
        		\tkzDrawPoints[fill=gray](A,B,C,D,E,F,G)
	\end{tikzpicture}
	\caption{Isosceles Triangle Theorem: conclusion}
\end{figure}
    
\end{proof}

\clearpage
