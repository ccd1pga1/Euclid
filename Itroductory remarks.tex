INTRODUCTORY REMARKS.
THE subject of Plane Geometry is here presented to the
student arranged in six books, and each book is subdivided
into propositions. The propositions are of two kinds, problems and theorems. In a problem something is required
to be done; in a theorem some new principle is asserted to be true.
A proposition consists of various parts. We have first
the general enunciation of the problem or theorem ; as for
example, To describe an equilateral triangle on a given
finite straight line, or Any two angles of a triangle are
together less than two right angles. After the general
enunciation follows the discussion of the proposition. First,
the enunciation is repeated and applied to the particular
figure which is to be considered ; as for example, Let AB
be the given straight line : it is required to describe an
equilateral triangle on AB. The construction then usually
follows, which states the necessary straight lines and circles
which must be drawn in order to constitute the solution of
the problem, or to furnish assistance in the demonstration
of the theorem. Lastly, we have the demonstration itself,
which shews that the problem has been solved, or that the
theorem is true.
Sometimes, however, no construction is required ; and
sometimes the construction and demonstration are combined.
xvi INTRODUCTORY REMARKS.
The demonstration is a process of reasoning in which
we draw inferences from results already obtained. These
results consist partly of truths established in former propositions, or admitted as obvious in commencing the subject,
and partly of truths which follow from the construction
that has been made, or which are given in the supposition
of the proposition itself. The word hypothesis is used in
the same sense as supposition.
To assist the student in following the steps of the
reasoning, references are given to the results already obtained which are required in the demonstration. Thus I. 5
indicates that we appeal to the result established in the
fifth proposition of the First Book ; Constr. is sometimes
used as an abbreviation of Construction, and Hyp. as an
abbreviation of Hypothesis.
It is usual to place the letters Q.E.F. at the end of the
discussion of a problem, and the letters Q.E.D. at the end of
the discussion of a theorem. Q.E.F. is an abbreviation for
quod erat faciendum, that is, which was to be done; and
Q.E.D. is an abbreviation for quod erat demonstrandum,
that is, which was to be proved.

%%%%%%%%%%%%%%
%%%%%%%%%%%%%%

Lewis Carroll, known by his real name Charles Lutwidge Dodgson, was a mathematician and logician as well as the author of "Alice's Adventures in Wonderland." In his work "Euclid and His Modern Rivals," published in 1879, Carroll presents a defense of Euclid's "Elements" as the preferred geometry textbook of the time, engaging in a fictional dialogue to critique contemporary geometry texts and to argue for the superiority of Euclid's approach to teaching geometry.

In "Euclid and His Modern Rivals," while Carroll discusses various aspects of Euclidean geometry and critiques many definitions, postulates, and propositions from both Euclid and other geometry texts of his time, his critique is more focused on the structure, pedagogy, and logical flow of Euclid's work compared to its rivals, rather than on specific definitions such as "The extremities of a surface are lines."

Carroll was keenly interested in the clarity and effectiveness of mathematical and logical education. His commentary often focused on how geometric principles were presented to learners, the rigor of logical arguments, and the importance of not assuming too much knowledge or intuition on the part of students. He valued Euclid's work for its systematic approach and clear, logical progression from simpler to more complex concepts.

Although Carroll provided detailed examinations and critiques throughout "Euclid and His Modern Rivals," there isn't a specific focus on Definition 6 of Book I about the extremities of a surface being lines in his critique. His work, however, offers valuable insights into the teaching of geometry during his time and highlights the enduring relevance and pedagogical value of Euclid's "Elements."

Carroll's engagement with Euclidean geometry through this work underscores his belief in the importance of a solid foundation in logic and clear reasoning in education, principles that are reflected in his critiques and advocacy for Euclid's methodology.