\definsection{Definition XXII}

\begin{defin}
\textgreek{Περὶ τετραγώνων σχημάτων, τετράγωνος μὲν ὁ ἴσα πλευρὰς ἔχων καὶ ὀρθογώνιος, ὀρθογώνιος δὲ ὁ μὲν ὀρθογώνιος ἔχων μὴ δὲ ἴσας πλευρὰς, ῥόμβος δὲ ὁ μὲν ἴσας πλευρὰς ἔχων μὴ δὲ ὀρθογώνιος, ρόμβοειδὴς δὲ ὁ ἀλλήλοις ἴσας ἔχων πλευρὰς καὶ γωνίας μὴ δὲ ὀρθογώνιος μηδ' ἴσας πλευράς. Τραπέζιον δὲ λεγέσθω περὶ τὰς πλὴν τούτων τετραγώνους.}

Of quadrilateral figures, a square is that which is both equilateral and right-angled; an oblong that which is right-angled but not equilateral; a rhombus that which is equilateral but not right-angled; and a rhomboid that which has its opposite sides and angles equal to one another but is neither equilateral nor right-angled. And let quadrilaterals other than these be called trapezia.
\end{defin}

In this context:

\begin{figure}[h]
	\centering
		\begin{subfigure}{0.3\textwidth}
			\begin{tikzpicture}[scale=0.5]
				\tkzDefPoint(0,0){A}
				\tkzDefPoint(3,0){B}
				\tkzDefPoint(0,3){C}
				\tkzDefPoint(3,3){D}
				\tkzDrawPolygon(A,B,D,C)
				\tkzDrawPoints(A,...,D)
				\tkzMarkRightAngles(A,B,C B,D,C D,C,A C,A,B)
			\end{tikzpicture}
			\sidecaption[a][-2cm]{a square}
		\end{subfigure}
		\begin{subfigure}{0.3\textwidth}
			\begin{tikzpicture}[scale=0.5]
					\tkzDefPoint(0,0){A}
				\tkzDefPoint(5,0){B}
				\tkzDefPoint(0,3){C}
				\tkzDefPoint(5,3){D}
				\tkzDrawPolygon(A,B,D,C)
				\tkzDrawPoints(A,...,D)
				\tkzMarkRightAngles(A,B,C B,D,C D,C,A C,A,B)
			\end{tikzpicture}
			\sidecaption[b][-1cm]{an oblong, also known as a rectangle.}
		\end{subfigure}
		\begin{subfigure}{0.3\textwidth}
			\begin{tikzpicture}[scale=0.5]
				\tkzDefPoint(0,0){A}
 				\tkzDefPoint(3,1){B}
 				\tkzDefPoint(1,3){C}
 				\tkzDefPoint(4,4){D}
 				\tkzDrawPolygon(A,B,D,C)
 				\tkzDrawPoints(A,...,D)
 				\tkzMarkSegment[mark=None,size=0.5cm, postaction = {decorate, decoration={markings,mark = at position .5 with {\arrow{>}}}}](A,B)
 				\tkzMarkSegment[mark=None,size=0.5cm, postaction = {decorate, decoration={markings,mark = at position .5 with {\arrow{>}}}}](C,D)
				\tkzMarkSegment[mark=None,size=0.5cm, postaction = {decorate, decoration={markings,mark = at position .5 with {\arrow{>>}}}}](A,C)
				\tkzMarkSegment[mark=None,size=0.5cm, postaction = {decorate, decoration={markings,mark = at position .5 with {\arrow{>>}}}}](B,D)
			\end{tikzpicture}
			\sidecaption{a rhombus.}
		\end{subfigure}
		
		\begin{subfigure}{0.4\textwidth}
			\begin{tikzpicture}[scale=0.5]
				\tkzDefPoint(0,0){A}
				\tkzDefPoint(5,0){B}
				\tkzDefPoint(2,3){C}
				\tkzDefPoint(4,3){D}
				\tkzDrawPolygon(A,B,D,C)
 				\tkzDrawPoints(A,...,D)
 				\tkzMarkSegment[mark=None,size=0.5cm, postaction = {decorate, decoration={markings,mark = at position .5 with {\arrow{>}}}}](A,B)
 				\tkzMarkSegment[mark=None,size=0.5cm, postaction = {decorate, decoration={markings,mark = at position .5 with {\arrow{>}}}}](C,D)
			\end{tikzpicture}
			\sidecaption[d][-1.5cm]{a trapezium, also referred to as a trapeze or trapezoid.}
		\end{subfigure}
		\begin{subfigure}{0.4\textwidth}
			\begin{tikzpicture}[scale=0.5]
				\tkzDefPoint(0,0){A}
 				\tkzDefPoint(5,1){B}
 				\tkzDefPoint(1,3){C}
 				\tkzDefPoint(6,4){D}
 				\tkzDrawPolygon(A,B,D,C)
 				\tkzDrawPoints(A,...,D)
 				\tkzMarkSegment[mark=None,size=0.5cm, postaction = {decorate, decoration={markings,mark = at position .5 with {\arrow{>}}}}](A,B)
 				\tkzMarkSegment[mark=None,size=0.5cm, postaction = {decorate, decoration={markings,mark = at position .5 with {\arrow{>}}}}](C,D)
				\tkzMarkSegment[mark=None,size=0.5cm, postaction = {decorate, decoration={markings,mark = at position .5 with {\arrow{>>}}}}](A,C)
				\tkzMarkSegment[mark=None,size=0.5cm, postaction = {decorate, decoration={markings,mark = at position .5 with {\arrow{>>}}}}](B,D)
			\end{tikzpicture}
			\sidecaption[e][-0.5cm]{a parallelogram, though not explicitly defined here.}
		\end{subfigure}
\end{figure}

\clearpage

Among these figures, Euclid primarily utilizes the concept of a square. The other figure names might have been common during Euclid's time, inherited from earlier versions of the Elements, or possibly introduced later.

\begin{figure*}[H]
\Tree [.Quadrilaterals 
        [.Parallelograms 
            [.Rectangular square oblong ]
            [.Non-rectangular rhombus rhomboid ]
        ]
        [.Non-\\parallelograms 
            [.{Two sides\\ parallel \\ (trapezium)} isosceles trapezium scalene trapezium ]
            [.{No sides\\ parallel \\ (trapezoid)} ]
        ]
      ]
\end{figure*}

Euclid extensively employs the concept of parallelograms or parallelogrammic areas without providing a formal definition. It's evident that he refers to quadrilaterals with parallel opposite sides, encompassing rhombi and rhomboids as special cases. Additionally, instead of "oblong," Euclid employs the term "rectangle" or "rectangular parallelogram," which encompasses both squares and oblongs.

Squares and oblongs are defined to have right angles, meaning all four angles are right angles. While these definitions may seem brief, their intended meaning can be inferred from their usage. For instance, Proposition I.46 constructs a square, ensuring that all four angles are right angles, not just one of them.

Euclid might have considered quadrilaterals as less fundamental or as variations of more basic shapes. Triangles, for example, are the simplest polygon, and all other polygons can be divided into triangles. Circles hold a unique place in geometry, being shapes of constant distance from a center point, and they were of particular interest in the mathematics and philosophy of ancient Greece. This might explain why Euclid gave more foundational importance to circles and triangles, exploring their properties in greater detail through multiple definitions.

As Euclid has not yet defined parallel lines and does not anywhere define a parallelogram, he is not in a position to make the more elaborate classification of quadrilaterals attributed by Proclus to Posidonius and appearing also in Heron's Definitions.

Definition 22, by detailing the triangle as a three-sided figure, and extending the classification to polygons with an increasing number of sides, does more than categorize; it unveils a hierarchy and methodology in approaching geometric analysis. In this context, Euclid's focus on the number of angles or sides isn't merely a classification system but a reflection of the inherent complexity and diversity within geometrical forms. This systematization reveals an elegant universe of geometry where forms are understood not just by their appearance but by their fundamental characteristics.

\clearpage