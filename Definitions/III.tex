\definsection{Definition III}

\begin{defin}
\textgreek{Οἱ τῶν γραμμῶν τετμημένοι συμπίπτουσιν ἐν τοῖς ἄκροις αὐτῶν σημείοις.}

"The intersections of lines and their extremities are points."
\end{defin}

This definition suggests that there's a relationship between certain lines and points, implying that a point can be considered an endpoint of a line. However, it doesn't provide a clear definition of what exactly "ends" are, nor does it specify how many ends a line can have. For example, while the circumference of a circle has no ends, a finite line segment has two distinct endpoints.

\fs

\begin{figure}[H]
    \centering
    \begin{subfigure}{0.4\textwidth}
        \begin{tikzpicture}
            \tkzDefPoint(0,0){A}
            \tkzDefPoint(3,0){B}
            \tkzDrawSegment(A,B)
            \tkzDrawPoints[red](A,B)
        \end{tikzpicture}
        \sidecaption[a][-2cm]{Points teminating a line segment}
    \end{subfigure}
    \begin{subfigure}{0.4\textwidth}
        \begin{tikzpicture}
            \tkzDefPoint(0,0){A}
            \tkzDefPoint(3,0){B}
            \tkzDefPoint(1,1){C}
            \tkzDefPoint(2,-1){D}
            \tkzInterLL(A,B)(C,D)   \tkzGetPoint{E}
            \tkzDrawSegments(A,B C,D)
            \tkzDrawPoints[red](A,B,C,D,E)
        \end{tikzpicture}
        \sidecaption[b][-1cm]{point on an intersection of two line segments}
    \end{subfigure}
    \caption{Points and intersections}
\end{figure}

\clearpage