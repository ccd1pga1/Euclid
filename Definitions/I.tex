\definsection{Definition I}

\begin{defin}
\textgreek{Σημεῖόν ἐστιν ὃ μέρος οὐκ ἔχει}.

"A point is that which has no part."
\end{defin}

A point, as defined by Euclid, is an entity without any dimensionality; it is the most basic element in geometry. In Euclidean geometry, points are fundamental building blocks from which all other geometric figures are constructed. Conceptually, a point has no size, no shape, and no extent,that is, it has neither length, breadth, nor thickness\margincite{Euclid}; it is simply a precise location in space.

The significance of Euclid's definition of a point lies in its foundational role in geometry. Points serve as the starting point for defining other geometric objects, such as lines, circles, and polygons. Without points, geometric reasoning and construction would be impossible.

Euclid's definition of a point has stood the test of time, forming the basis of classical geometry for over two millennia. Mathematicians and geometers have recognized the elegance and simplicity of this definition, which 
encapsulates the essence of spatial location without unnecessary complexity.

Many mathematicians and philosophers throughout history have refered to or acknowledged Euclid's definition of a point. 
For example, Rene Descartes, in his development of Cartesian coordinates, relied on the concept of points as fundamental entities in space. Additionally, mathematicians such as David Hilbert and Bertrand Russell have discussed the foundational importance of points in geometry and set theory, further reinforcing the enduring significance of Euclid's definition.

\clearpage

In modern mathematics, the concept of a point extends beyond Euclidean geometry to various mathematical contexts, including abstract algebra, topology, and analysis. Despite these advancements, Euclid's definition of a point remains a cornerstone of geometric reasoning and continues to inspire mathematical exploration and discovery.


\clearpage