\definsection{Definition II}

\begin{defin}
\textgreek{Γραμμή ἐστι μῆκος ἀπλατὸν.}

"A line is breadthless length."
\end{defin}

Euclid continues to present his mathematical concepts and proofs in a rigorous and systematic manner. In the context of Definition 2 of Book 1, Euclid intended "straight line segment" to represent the shortest distance between two points in a plane. In Euclidean geometry, a straight line is defined as the path traced by a moving point that remains consistently equidistant from two fixed points. Euclid's definition was clear in its intent to describe this fundamental geometric concept, and it served as the basis for subsequent mathematical developments.

\fs

\sidenote{\small{An infinite straight line segment.}}
\begin{figure}[H]
\begin{centering}
	\begin{tikzpicture}
		\tkzDefPoint(0,0){A}
		\tkzDefPoint(3,0){B}
		\tkzDrawSegment[red](A,B)
	\end{tikzpicture}
	\caption{A line}
\end{centering}
\end{figure}

While Euclid's geometry primarily deals with straight lines, curves and other geometric figures were considered in later mathematical developments. However, within the scope of Euclid's Elements, the focus is on straight lines and their properties, including being the shortest distance between two points. Any ambiguity regarding the definition of lines as curves is more of a modern consideration, as Euclid's approach was firmly rooted in the concept of straightness.


\clearpage