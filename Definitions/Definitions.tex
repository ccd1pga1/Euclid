
\section{Definitions}

\begin{enumerate}

\item A point is that which has no part.
\item A line is breadthless length.
\item The extremities of a line are points.
\item A straight line is a line which lies evenly with the
points on itself.
\item A surface is that which has length and breadth only.
\item The extremities of a surface are lines.
\item A plane surface is a surface which lies evenly with
the straight lines on itself.
\item A plane angle is the inclination to one another of
two lines in a plane which meet one another and do not lie in a straight line.
\item And when the lines containing the angle are straight, the angle is called rectilineal.
\item When a straight line set up on a straight line makes the adjacent angles equal to one another, each of the equal angles is right, and the straight line standing on the other is called a perpendicular to that on which it stands.
\item An obtuse angle is an angle greater than a right angle.
\item An acute angle is an angle less than a right angle.
\item A boundary is that which is an extremity of any­
thing.
\item A figure is that which is contained by any boundary or boundaries.
\item A circle is a plane figure contained by one line such that all the straight lines falling upon it from one point among those lying within the figure are equal to one another;
\begin{itemize}
\item And the point is called the centre of the circle.
\end{itemize}
\item A diameter of the circle is any straight line drawn through the centre and terminated in both directions by the circumference of the circle, and such a straight line also bisects the circle.
\item A semicircle is the figure contained by the diameter and the circumference cut off by it. And the centre of the semicircle is the same as that of the circle.
\item Rectilineal figures are those which are contained by straight lines, trilateral figures being those contained by three, quadrilateral those contained by four, and multi­ lateral those contained by more than four straight lines.
\item Of trilateral figures, an equilateral triangle is that which has its three sides equal, an isosceles triangle that which has two of its sides alone equal, and a scalene triangle that which has its three sides unequal.
\item Further, of trilateral figures, a right-angled tri­ angle is that which has a right angle, an obtuse-angled triangle that which has an obtuse angle, and an acute- angled triangle that which has its three angles acute.
\item Of quadrilateral figures, a square is that which is both equilateral and right-angled ; an oblong that which is right-angled but not equilateral; a rhombus that which is equilateral but not right-angled ; and a rhomboid that which has its opposite sides and angles equal to one another but is neither equilateral nor right-angled. A n d let quadrilaterals other than these be called trapezia.
\item Parallel straight lines are straight lines which, being in the same plane and being produced indefinitely in both directions, do not meet one another in either direction.

\end{enumerate}

\clearpage