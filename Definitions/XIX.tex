\definsection{Defiition XIX}

\begin{defin}
"Rectilinear figures are those which are contained by straight lines, trilateral figures being those contained by three, quadrilateral those contained by four, and multilateral those contained by more than four straight lines."
\end{defin}

\begin{figure}[H]
\begin{centering}
	\begin{subfigure}{0.3\textwidth}
		\begin{tikzpicture}
			\tkzDefPoint(0,0){A}
			\tkzDefPoint(1,3){B}
			\tkzDefPoint(2.8,2){C}
			\tkzDrawPolygon[fill=blue, opacity= 0.1](A,B,C)
		\end{tikzpicture}
		\sidecaption[a][-2cm]{a triangle}
	\end{subfigure}
	\hfill
	\begin{subfigure}{0.3\textwidth}
		\begin{tikzpicture}
			\tkzDefPoint(0,0){A}
			\tkzDefPoint(1,2){B}
			\tkzDefPoint(-0.5,3){C}
			\tkzDefPoint(2.8,1.5){D}
			\tkzDrawPolygon[fill=green, opacity= 0.1](A,B,C,D)
		\end{tikzpicture}
		\sidecaption[b][-1cm]{a quadrilateral, or tetragon}
	\end{subfigure}
	\hfill
	\begin{subfigure}{0.3\textwidth}
		\begin{tikzpicture}
			\tkzDefPoint(0,0){A}
			\tkzDefPoint(0.5,1){B}
			\tkzDefPoint(0.3,2){C}
			\tkzDefPoint(2,1.5){D}
			\tkzDefPoint(2.8,1.9){E}
			\tkzDefPoint(2.3,2){F}
			\tkzDefPoint(2.8,0.2){G}
			\tkzDefPoint(2,1){H}
			\tkzDrawPolygon[fill=red, opacity= 0.1](A,B,C,D,F,E,G,H)
		\end{tikzpicture}
		\sidecaption{An octagon}
	\end{subfigure}
\end{centering}
\caption{Polygons}
\end{figure}


In Euclid's "Elements," a foundational text in the study of geometry, Definition 19 serves as a critical juncture in the exploration of geometric figures. This definition, which introduces rectilinear figures as entities enclosed by straight lines, may seem simplistic at first glance. However, it marks a significant shift in Euclidean geometry from the abstract principles of space and shape to their practical manifestations.

Euclid's approach to classification predominantly focuses on the angles of figures rather than their sides. This method is evident in his construction and naming of regular polygons in Book IV, where figures are identified by the number of their angles—such as pentagons (five-angled figures), hexagons (six-angled figures), and even pentadecagons (fifteen-angled figures). This angle-centric nomenclature is largely consistent with modern naming conventions for polygons, which also emphasize the number of angles, except in the case of "quadrilaterals," where the classification is based on sides. It's worth noting that while the names for polygons from "triangle" to "octagon" are derived from Greek, the usage of specific terms for figures with more than eight sides is uncommon in everyday practice. Additionally, the term "quadrilateral" is sometimes replaced with "tetragon," though the former is more prevalent.

Contrary to what might be implied by the straightforward definition of rectilinear figures, Definition 19 encapsulates a profound philosophical and methodological underpinning of Euclidean geometry. It represents a deliberate transition from abstract geometric concepts to their concrete counterparts, mirroring how the center of a circle acts as a fulcrum for understanding its geometric properties. The delineation of a figure by straight lines is not merely a matter of definition but a foundational principle that sets the stage for the exploration of geometric relationships and properties. In this light, Euclid's meticulous detailing of boundaries and figures underscores the rigor and systematic approach that characterizes Euclidean geometry. Every element, regardless of its apparent simplicity, is integral to the cohesive understanding of geometric principles.

This meticulous approach by Euclid, emphasizing the importance of foundational definitions and classifications, underscores the timeless relevance of "Elements" in the study of geometry. By establishing clear and precise definitions, Euclid not only facilitated a deeper understanding of geometric figures but also laid the groundwork for the systematic exploration of space and form. Definition 19, in particular, exemplifies the interplay between the abstract and the tangible, highlighting the nuanced and thoughtful methodology that defines Euclidean geometry.



\clearpage