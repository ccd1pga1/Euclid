\definsection{Definition X}

\begin{defin}
\textgreek{Ορθή Γωνία}

When a straight line set up on a straight line makes the adjacent angles equal to one another, each of the equal angles is right, and the straight line standing on the other is called a perpendicular to that on which it stands.
\end{defin}

In the geometry of Euclid, a right angle serves as a fundamental building block, defined with simplicity yet profound in its implications. This angle is created when a straight line stands on another straight line, making the adjacent angles equal. Such an angle is not just any angle but a right angle, marking the epitome of equality and perpendicularity in geometric relations.

\sidenote{\ssmall{Although not precise enought for Euclid we wouold desribe  right angle is an angle of exactly $\ang{90}$, forming a perfect L shape. }}
\begin{figure}[H]
	\centering
	\begin{tikzpicture}
		\tkzDefPoint(0,0){A}
		\tkzDefPoint(3,0){B}
		\tkzDefPoint(1.5,0){C}
		\tkzDefPoint(1.5,2){D}
		\tkzDrawSegments (A,B C,D)
		\tkzMarkRightAngle[thick,red](A,C,D)
		\tkzMarkRightAngle[thick, blue](B,C,D)
	\end{tikzpicture}
	\caption{A Perpendiuular line}
\end{figure}

This definition does more than just describe; it establishes a cornerstone upon which much of Euclidean geometry rests. The right angle's properties are pivotal in constructing squares, rectangles, and understanding the principles that underpin the Pythagorean theorem. Its introduction early in Euclid's Elements is a testament to its foundational role in geometry. As we explore this concept, we see not just a definition but a gateway to understanding the spatial relationships that are central to the discipline. The clarity and precision with which Euclid delineates this term reflect a methodology that values rigor and simplicity, guiding learners from basic principles to complex constructions with logical elegance.