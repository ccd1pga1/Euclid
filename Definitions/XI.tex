\definsection{Definition XI}

\begin{defin}
\textgreek{Οξεία Γωνία}

An obtuse angle is an angle greater than a right angle.
\end{defin}

Moving from the foundational right angle, Euclid introduces the obtuse angle—a concept that expands our geometric vocabulary and understanding. An obtuse angle is one that is greater than a right angle. This definition, though brief, opens up a new dimension of angular measurement and comparison.

\sidenote{\ssmall{Again not for Euclide but, an obtuse angle is an angle greater than \ang{90} but less than \ang{180}.}}
\begin{figure}[H]
	\centering
	\begin{tikzpicture}
		\tkzDefPoint(0,0){A}
		\tkzDefPoint(3,0){B}
		\tkzDefPoint(1.5,0){C}
		\tkzDefPoint(0.8,1.8){D}
		\tkzDrawSegments(A,B C,D)
		\tkzMarkAngle[red](B,C,D)
	\end{tikzpicture}
	\caption{An obtuse angle}
\end{figure}

The significance of obtuse angles extends far beyond their simple definition. They play a crucial role in the classification of triangles, contributing to our understanding of the diverse geometric shapes and their properties. Obtuse angles challenge the learner to think about angles in a comparative manner, laying the groundwork for more advanced studies in trigonometry and the analysis of geometric figures.

Euclid's separate treatment of obtuse angles underscores the importance of nuanced distinctions in geometry. By differentiating between right, obtuse, and acute angles, Euclid ensures that learners grasp the full spectrum of angular possibilities, enriching their geometric comprehension and problem-solving capabilities. This approach exemplifies the educational philosophy of building knowledge step by step, ensuring a deep and lasting understanding.