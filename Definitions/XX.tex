\definsection{Definition XX}

\begin{defin}
"Of trilateral figures, an equilateral triangle is that which has its three sides equal, an isosceles triangle that which has two of its sides alone equal, and a scalene triangle that which has its three sides unequal."
\end{defin}

Definition 20 classifies triangles based on their symmetries,

According to Definition 20:

A scalene triangle (C) has no symmetries.
An isosceles triangle (B) has bilateral symmetry.
An equilateral triangle (A) not only possesses three bilateral symmetries but also 120° rotational symmetries.

\begin{figure}[h]
\begin{subfigure}{0.305\textwidth}
\centering
\begin{tikzpicture}
\ETri{0,0}{A}{B}{C}
\tkzDefLine[altitude](A,B,C)		\tkzGetPoint{D}
\tkzDrawSegment[dashed,red](B,D)
\tkzDefLine[altitude](B,A,C)		\tkzGetPoint{E}
\tkzDrawSegment[dashed,red](A,E)
\tkzDefLine[altitude](B,C,A)		\tkzGetPoint{F}
\tkzDrawSegment[dashed,red](C,F)
\end{tikzpicture}
\sidecaption[a][-4cm]{\raggedright\ssmall{An equilateral triangle possesses three bilateral symmetries and also $\ang{120}$ rotational symmetries.}}
\end{subfigure}
%%%%%%%%
\begin{subfigure}{0.35\textwidth}
\begin{tikzpicture}
\tkzDefPoint(0,1){A}
\tkzDefPoint(3,1){B}
\tkzDefPoint(1.5,3.5){C}
\tkzDrawPolygon(A,B,C)
\tkzMarkSegment[mark=|](A,C)
\tkzMarkSegment[mark=|](B,C)
\tkzDefLine[altitude](A,C,B)		\tkzGetPoint{D}
\tkzDrawSegment[dashed,red](C,D)
\end{tikzpicture}
\sidecaption[b][-2cm]{\raggedright\ssmall{An isosceles triangle has bilateral symmetry.}}
\end{subfigure}
%%%%%%%%%%
\begin{subfigure}{0.31\textwidth}
\begin{tikzpicture}
\Tri{A}{2.5}{-1.5}{B}{1}{2}{C}{none}
\end{tikzpicture}
\sidecaption[c][-1cm]{\raggedright\ssmall{A scalene triangle has no symmetries.}}
\end{subfigure}
\caption{Triangles and their symmetry}
\end{figure}


It's worth noting that under this definition, an equilateral triangle is not considered an isosceles triangle. However, in Euclid's Elements, the term "isosceles triangle" is introduced in $\textcolor{red}{Proposition I.5,}$ and later in $\textcolor{red}{Books II}$ and $\textcolor{red}{IV}$. The usage of "isosceles triangle" in the Elements does not exclude equilateral triangles. In modern practice, it's only necessary for at least two sides to be equal for a triangle to be classified as isosceles.

\clearpage

Equilateral triangles are constructed in the first proposition of the $\textcolor{red}{Elements, I.1}$. Additionally, an alternate characterization of isosceles triangles, namely that their base angles are equal, is demonstrated in $\textcolor{red}{Propositions I.5}$ and $\textcolor{red}{I.6}$.


\clearpage