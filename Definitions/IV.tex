\definsection{Definition IV}

\begin{defin}
\textgreek{Γραμμή δὲ ἡ μεσαία ἐνδιάμεσον ἐῶσα τὰ ἄκρα ἔχει, καλεῖται εὐθεία ἢ ὀρθὴ γραμμή, οἷον ἡ ΑΒ.}

A line which lies evenly between its extreme points is called a straight or right line, such as $\overline{AB}$.
\end{defin}

This was not as easy as it seams for Euclid to define,  \margincite{Heath}The truth is that Euclid was attempting the impossible. As Pfleiderer says (Scholia to Euclid), " It seems as though the notion of a straight line, owing to its simplicity, cannot be explained by any regular definition which does not introduce words already containing in themselves, by implication, the notion to be defined (such e.g. are direction, equality, uniformity or evenness of position, unswerving course), and as though it were impossible, if a person does not already know what the term straight here means, to teach it to him unless by putting before him in some way a picture or a drawing of it."

If a point moves without changing its direction it will describe a straight (or right line). The direction in which a point moves in called its “sense.” If the moving point continually changes its direction it will describe a curve; hence it follows that only one right line can be drawn between two points.

\sidenote{\small{A finite stright line with distictive ends.}}
\begin{figure}[h]
	\begin{centering}
		\begin{tikzpicture}
			\tkzDefPoint(0,0){A}
			\tkzDefPoint(3,0){B}
			\tkzDrawSegment(A,B)
			\tkzLabelPoints(A,B)
			\tkzDrawPoints[red](A,B)
		\end{tikzpicture}
		\caption{A straight line}
	\end{centering}
\end{figure}

“If we suspend a weight by a string, the string becomes stretched, and we say it is straight, by which we mean to express that it has assumed a peculiar definite shape. If we mentally abstract from this string all thickness, we obtain the notion of the simplest of all lines, which we call a straight line."\margincite{Henrici}

The straight line is foundational to geometry, serving as a fundamental object upon which many geometric concepts and constructions are based. Euclid's treatment of straight lines in his Elements primarily revolves around Book I, where he lays down the foundational definitions and postulates.

However, the simplicity of this definition belies the richness of the concept of a straight line. Throughout history, mathematicians have grappled with the notion of straightness, leading to various interpretations and misunderstandings.

One notable example is the parallel postulate, which states that given a line and a point not on that line, there exists exactly one line parallel to the given line through the given point. Euclid included this postulate as one of his five postulates, but its uniqueness and seemingly independent nature led mathematicians to question its validity and explore alternatives.

In the 19th century, mathematicians such as Nikolai Lobachevsky and János Bolyai challenged the assumption of Euclidean geometry by developing non-Euclidean geometries where the parallel postulate does not hold true. Their work paved the way for the development of hyperbolic geometry, where straight lines behave differently than in Euclidean geometry.

Later, in the early 20th century, Albert Einstein's theory of general relativity further revolutionized our understanding of straight lines. In the context of curved spacetime, straight lines represent the paths that objects follow in the presence of gravitational fields. These "geodesics" are not necessarily Euclidean straight lines but rather the shortest paths between points in curved spacetime.

In summary, while Euclid's definition of a straight line laid the groundwork for classical geometry, subsequent mathematicians have expanded and challenged this concept, leading to new understandings and interpretations in both Euclidean and non-Euclidean geometries, as well as in the context of modern physics.

\clearpage